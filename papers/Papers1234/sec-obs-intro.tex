
\section{Introduction}
\label{sec:introduction}

Stellar bow shocks are produced by the relative motion between a star
and its surrounding medium, and are commonly detected as curved arcs
of emission at optical \citep{Gull:1979a, Brown:2005a}, infrared
\citep{van-Buren:1988a, Kobulnicky:2016a}, or radio
\citep{van-Buren:1990a, Benaglia:2010a} wavelengths.  The canonical
theory for these objects is that they are formed by a two-shock
interaction between the stellar wind and the interstellar medium
\citep{Pikelner:1968a, Dyson:1972a}, which is distorted due to the
supersonic motion of the star \citep{Baranov:1970a, Wilkin:1996a}.  In
some instances, however, the absorbed stellar radiation pressure may
be more important than the stellar wind in providing the inner support
for the bow shell \citep[Paper~I]{Henney:2019a} and this may even be
sufficient to break the collisional coupling between gas and dust
grains \citep[Paper~II]{Henney:2019b}.  In other cases, the appearance
of an infrared emission arc may be due to the illumination of the
inner wall of an asymmetrical cavity \citep{Mackey:2016a}, rather than
the formation of a dense shell, in which case the relative velocity of
the star may be subsonic with respect to its surroundings
\citep{Mackey:2015a}.

The largest number of bow shocks have been detected around OB stars,
via their mid-infrared dust emission \citep{van-Buren:1995a,
  Smith:2005a, Povich:2008a, Kobulnicky:2010a, Peri:2012a, Peri:2015a,
  Sexton:2015b, Kobulnicky:2016a, Bodensteiner:2018a}, and these have
sizes ranging from \SIrange{0.01}{1}{pc}. 

Wolf Rayet stars give bigger bow shocks (e.g., WR~128 \citealp{Moffat:1998})

In the Orion Nebula, at least three different classes of stellar bow
shocks have been identified. As well as small number of OB bow shocks \citep{Smith:2005a, ODell:2001c}, 

Interaction of wind with photoevaporation flow \citep{Dyson:1975a}

HMXRB in external galaxies, possible bowshock in LMC~X-1 \citep{Hyde:2017a}.

Wolf-Rayet bow shock nebulae \citep{Dyson:1989a}

Non-thermal radio emission from BD+43~3654 \citep{Benaglia:2010a}, has
been searched for but

Cooler red supergiant and asymptotic giant branch stars
\citep{Ueta:2008a, Sahai:2010a, Cox:2012a}. 
Bowshocks from red supergiants \citep{Meyer:2014a}


Instability of bow \citep{Blondin:1998a}

%%% Local Variables:
%%% mode: latex
%%% TeX-master: "obs-bowshocks"
%%% End:
