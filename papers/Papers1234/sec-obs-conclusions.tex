
\section{Conclusions}
\label{sec:conclusion}

Difference in SEDs between FIR and MIR samples?  \citet{Meyer:2016a}
say emission peaks at 3--50 micron for O star bow shocks.

Al other things being equal, larger bows will be at higher
inclinations (we should estimate how much variation in size we should
get due to inclination -- it will be more for flatter bows).  Higher
inclinations means less variation from the standing wave oscillations,
which goes against our result that larger bows have greater dispersion
in \(\Lambda'\).  Although the standing waves mainly give dispersion in
\(\Pi'\) anyhow.

\citet{Meyer:2014a} say that in cool stars the dust emission comes
from the shocked inner wind, rather than from near the contact
discontinuity for hot stars.  This could explain the difference in
shape between the Herschel and Spitzer samples.


How different regions of the \(\Pi\)--\(\Lambda\) plane are populated.
Bottom-right quadrant hard to get to (except for standing wave
oscillations), but may be due to finite shell thickness, which (for
low Mach number) will be more apparent in the wings, which might
decrease \(\Lambda\) more than \(\Pi\).  Fact that thin-shell solutions should
trace the contact discontinuity, but in some cases it may be only the
inner or the outer shell that is visible.

Justification for standing waves: Fig.~3 of \citet{Meyer:2016a} shows
a time sequence of thin-shell instability, which looks a bit like a
standing wave. But much larger amplitude than we are considering.

Deviations from axisymmetry as an alternative to oscillations. 

\subsection{The case of inside-out bows}
\label{sec:case-inside-out}

So far, we have considered the case where the inner source dominates
the radiation, while dust is present only in the outer stream, which
applies to hot stars interacting with the ISM.  However, in the case
of cool stars, the inner wind will also be dusty.  Examples are the
red supergiant (RSG) phase of high-mass evolution, or the asymptotic
giant branch (AGB) stage of low/intermediate-mass evolution.  In both
these cases, it is still the inner source that provides the radiation
field.  However, not all winds are radiatively driven and in those
cases it is conceivable that it is the outer source that dominates the
radiation field.  An example is the case of photoevaporating
protoplanetary disks (proplyds) in the Orion Nebula and other \hii{}
regions \citep{ODell:1994a}.  In the proplyds, the inner wind is a
thermally driven photoevaporation flow \citep{Henney:1998b, Henney:1999a},
while the outer stream is the stellar wind from an O~star
\citep{Garcia-Arredondo:2001a}.


%%% Local Variables:
%%% mode: latex
%%% TeX-master: "obs-bowshocks"
%%% End:
