
\section{Conclusions}
\label{sec:conclusion}

Difference in SEDs between FIR and MIR samples?  \citet{Meyer:2016a}
say emission peaks at 3--50 micron for O star bow shocks.

Al other things being equal, larger bows will be at higher
inclinations (we should estimate how much variation in size we should
get due to inclination -- it will be more for flatter bows).  Higher
inclinations means less variation from the standing wave oscillations,
which goes against our result that larger bows have greater dispersion
in \(\Lambda'\).  Although the standing waves mainly give dispersion in
\(\Pi'\) anyhow.

\citet{Meyer:2014a} say that in cool stars the dust emission comes
from the shocked inner wind, rather than from near the contact
discontinuity for hot stars.  This could explain the difference in
shape between the Herschel and Spitzer samples.

%%% Local Variables:
%%% mode: latex
%%% TeX-master: "obs-bowshocks"
%%% End:
