\section{Case studies}
\label{sec:case-studies}

Carina mid-infrared bow shocks \citep{Sexton:2015b} are in a high
density environment, \SI{1000}{cm^{-3}}, so they may be bow waves.
There seems to be spectral types for most of them: B0 (but supergiant)
to O7.  Sizes are 3 to 12 arcsec, which at Carina (\SI{2.3}{kpc}) is
\SIrange{0.033}{0.134}{pc}.

Amazingly, the size/density combination gives regions that overlap
with the dust wave region for both the \SI{20}{M_\odot} and \SI{40}{M_\odot}
case.  And implying velocities of \SIrange{30}{50}{km.s^{-1}}.  This
is more believable than the \SI{10}{km.s^{-1}} that they quote, since
that would not give a shock at all.  This would imply
\(\tau > \eta \approx 0.1\), so the bow luminosity should be 10\% of the star
luminosity, so getting on for \SI{e4}{L_\odot}.

Two small bows in M42:

\(\theta^1\)\,Ori~D (Ney--Allen nebula) \citep{Robbertp:2005a}

LP~Ori: B1.5V star, like our \SI{10}{M_\odot} example. Radius about
\(3''\), so 0.005 pc.  With \(v = 80\) that would clearly be a dust
wave and would require \SIrange{100}{1000}{cm^{-3}}. Gaia distance \SI{408 +- 11}{pc}


Ones that Ochsendorf claims are dust waves (Narrator: they aren't).


%%% Local Variables:
%%% mode: latex
%%% TeX-master: "dusty-bow-wave"
%%% End:
