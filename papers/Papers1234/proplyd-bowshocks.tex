
\RequirePackage{amsmath}
\documentclass[useAMS, usenatbib]{mnras}

% The following is needed to fix the margins if using Letter-size paper
% REMOVE if your LaTeX uses A4 paper by default
%\addtolength\topmargin{-1.8cm}

% Standard LaTeX packages
% \usepackage[varg]{txfonts}
\usepackage[varg]{newtxmath}
\usepackage{newtxtext}
\usepackage{graphicx}
\usepackage{microtype}
\usepackage{xcolor}
\usepackage{fixltx2e}
\usepackage{booktabs}
\usepackage{hyperref}
\usepackage{siunitx}
\usepackage{color}
\usepackage{appendix}
\hypersetup{colorlinks=True, linkcolor=blue!50!black, citecolor=black,
  urlcolor=blue!50!black}

%% Bold italic
\newcommand\hmmax{0}            % we don't need heavy fonts
\newcommand\bmmax{1}            % reduce use of math alphabets for bold
\usepackage{bm}

%% Bundled custom packages
\usepackage{aastex-compat}
%\usepackage{astrojournals}


\title[Proplyd bowshocks]{What is confining the proplyd bow shocks in Orion?}

\newcommand\AddressCRyA{Instituto de Radioastronom\'{\i}a y Astrof\'{\i}sica,
  Universidad Nacional Aut\'onoma de M\'exico, Apartado Postal 3-72,
  58090 Morelia, Michoac\'an, M\'exico}
\author[Tarango Yong \& Henney]{
  Jorge A. Tarango Yong \& William J. Henney\\
  \AddressCRyA
}
\begin{document}
\maketitle
\begin{abstract}
  We analyze the projected shapes of bowshocks associated with
  photoevaporating disks (proplyds) in the inner Orion Nebula.  We
  show that the photoevaporation flows must be anisotropic in order to
  reproduce the observed shapes.  The stagnation pressures of the
  shocked shells within 0.02~pc of the dominant high-mass star are
  consistent with an interaction with its unshocked stellar wind, but
  the more distant shells require a confining pressure that is an
  order of magnitude higher than can be provided by the unmodified
  wind.  We show that these outer objects may be proplyds that are
  interacting with the conical wake generated by the wind-wind
  interaction in the inner proplyds. 
\end{abstract}

\newcommand\thC{\(\theta^1\)\,Ori~C}
\defcitealias{Canto:1996}{CRW}
\newcommand\CRW{\citetalias{Canto:1996}}


\section{Introduction}
\label{sec:introduction-proplyd}

Proplyds are comet-like structures observed in HII regions like Orion Nebula Cluster (ONC). 
These objects are interpreted as a D-type Ionization Front (IF) of a photoevaporated flow 
originated in the protoplanetary disk of a nearby low mass YSO \citep{Johnstone:1998}.
The pressure of the surrounding gas is not enough to confine this flow \citep{HA:1998}
may be formed by the interaction of the photoevaporated wind of the proplyds with the stellar wind of \thC{}, which is highly supersonic $(M \sim 100)$. 


%%% Local Variables:
%%% mode: latex
%%% TeX-master: "proplyd-bowshocks"
%%% End:

\section{Application to proplyd bowshocks}
\label{sec:application}

Proplyds are comet-like structures observed in HII regions like Orion Nebula Cluster (ONC). 
These objects are interpreted as a D-type Ionization Front (IF) of a photoevaporated flow 
originated in the protoplanetary disk of a nearby low mass YSO \citep{Johnstone:1998}.
The presure of the surrounding gas is not enough to confine this flow \citep{HA:1998}
may be formed by the interaction of the photoevaporated wind of the proplyds with the stellar wind of $\theta^1~C~Ori$, which is highly supersonic $(M \sim 100)$. 

The density distribution of the photoevaporated flow can be determined using the steady state continuity equation and assuming that almost all ionizing photons are absorbed at the IF \citep{HA:1998} and ignoring dust absorption, can be found that

\begin{align}
N(r_{IF},\theta) = N_0 \cos^{0.5}\theta
\label{eq:nprop}
\end{align}

This scenario can be described with the formalism of \citep{Canto:1996}, who proposes an algebraic solution for the two winds interaction problem in the thin shell approximation. See figure.

 Combining equations (6),(9)-(11) and (19)-(23) from \citep{Canto:1996} and (\ref{eq:nprop}), we can solve numerically for $R(\theta)$, as shown in figure (add figure later). Also we can
obtain a relation between $\theta$ and $\theta_1$

\begin{align}
\theta_1\cot\theta_1 = 1+ 2\beta I(\theta)\cot\theta - \frac{4}{5}\beta\left(1-\cos^{5/2}\theta\right)
\label{eq:th1th}
\end{align}

\subsection{Characteristic Radii}

$R_0$ is obtained directly from equation () of \citep{Canto:1996} as the distance from the inner source where the RAM pressure of the interacting winds is in equilibrium.

Combinig equation (23) from \citep{Canto:1996} and  (\ref{eq:th1th}) evaluated at $\theta=\frac{\pi}{2}$, we can obtain $R_{90}$ 

\begin{align}
R_{90} \simeq \frac{\left(2.4\beta\right)^{1/2}}{1-0.8\beta}
\label{eq:r90}
\end{align}
%%% Local Variables:
%%% mode: latex
%%% TeX-master: "proplyd-bowshocks"
%%% End:


\section{Conclusions}
\label{sec:conclusions-proplyd}

Evidence for the leaky sieve model for channeling the stellar wind
momentum.

Alternative would be nebular champagne flow from Orion S.  We need to
work out the numbers for that. But the orientation seems too close to
radial for that to be plausible.

We also need to show that radiation pressure cannot be important.

Make sure we analyse the distribution of inclinations, taking into
account there is a maximum inclination above which a bowshock will not
be seen. 


\bibliographystyle{mnras}
%% All references should be put in the BibTeX file bowshocks-biblio.bib
\bibliography{bowshocks-biblio}

\end{document}

%%% Local Variables:
%%% mode: latex
%%% TeX-master: t
%%% End:
