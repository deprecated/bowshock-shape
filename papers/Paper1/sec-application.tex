\section{Application to proplyd bowshocks}
\label{sec:application}

Proplyds are comet-like structures observed in HII regions like Orion Nebula Cluster (ONC). 
These objects are interpreted as a D-type Ionization Front (IF) of a photoevaporated flow 
originated in the protoplanetary disk of a nearby low mass YSO \citep{Johnstone:1998}.
The presure of the surrounding gas is not enough to confine this flow \citep{HA:1998}
may be formed by the interaction of the photoevaporated wind of the proplyds with the stellar wind of $\theta^1~C~Ori$, which is highly supersonic $(M \sim 100)$. 

The density distribution of the photoevaporated flow can be determined using the steady state continuity equation and assuming that almost all ionizing photons are absorbed at the IF \citep{HA:1998} and ignoring dust absorption, can be found that

\begin{align}
N(r_{IF},\theta) = N_0 \cos^{0.5}\theta
\end{align}

This scenario can be described with the algebraic formalism of \citep{Canto:1996}.
%%% Local Variables:
%%% mode: latex
%%% TeX-master: "proplyd-bowshocks"
%%% End:
