\section{Application to proplyd bowshocks}
\label{sec:application}

Proplyds are comet-like structures observed in HII regions like Orion Nebula Cluster (ONC). 
These objects are interpreted as a D-type Ionization Front of a photoevaporated flow 
originated in the protoplanetary disk of a nearby low mass YSO \citep{Johnstone:1998}.
The presure of the surrounding gas is not enough to confine this flow, thus accelerates as a 
supersonic wind slowly until $M \sim 3$. In the particular case of the ONC, the ``Classic'' proplyds
discovered by Laques \& Vidal \citep{Laques:1979}, which are spatially nearby $\theta^1~C~Ori$, an stationary bowshock
may be formed by the interaction of the photoevaporated wind of the proplyds with the stellar wind of $\theta^1~C~Ori$.
%%% Local Variables:
%%% mode: latex
%%% TeX-master: "proplyd-bowshocks"
%%% End:
