\section{Paraboloids and their plane-of-sky projection}
\label{app:parabola}

Equation~\eqref{eq:par-xy} for the \(xy\) coordinates of a quadric in the \(\phi = 0\) plane cannot be used in the case of a paraboloid (\(\Q = 0\)).  Instead, a convenient parametrization is
\begin{gather}
  \label{eq:parabola-xy}
  \begin{aligned}
    x &= R_0 \left(1  - \tfrac{1}{2} \Pi\, t^2\right) \\
    y &= R_0\, \Pi\, t \ ,
  \end{aligned}
\end{gather}
where we have ``baked in'' knowledge of the planitude,
\(\Pi = R_c/R_0\) (see \S~\ref{sec:plan-alat-bow}). The projected
plane-of-sky coordinates of the tangent line follow from
equation~\eqref{eq:Trans} as
\begin{gather}
  \label{eq:parabola-xy-prime-phi}
  \begin{aligned}
    x_{\T}' / R_0 &= \left(1 - \tfrac{1}{2} \Pi\, t^2\right) \cos i
      + \Pi\, t \sin\phi_{\T} \sin i\\
    y_{\T}' / R_0 &= \Pi\, t \cos\phi_{\T}\ ,
  \end{aligned}
\end{gather}
The azimuth of the tangent line is found from
equations~(\ref{eq:alpha}, \ref{eq:tanphi}) as
\(\sin\phi_{\T} = -t^{-1} \tan i \), so that
\begin{gather}
  \label{eq:parabola-xy-prime-final}
  \begin{aligned}
    x_{\T}' / R_0 &= \cos i \left[ 1 + \tfrac{1}{2} \Pi \tan^2 i -
      \tfrac{1}{2} \Pi \left( t^2 - \tan^2 i \right) \right]\\
    y_{\T}' / R_0 &= \Pi \left( t^2 - \tan^2 i \right)^{1/2} \ .
  \end{aligned}
\end{gather}
The projected star--apex distance, \(R_0'\), is the value of
\(x_{\T}'\) when \(y_{\T}' = 0\), yielding
\begin{equation}
  \label{eq:parabola-R0-prime}
  R_0' / R_0 = \cos i \left( 1 + \tfrac{1}{2} \Pi \tan^2 i  \right) \ . 
\end{equation}
Note that this same result can be obtained from a Taylor expansion of
equation~\eqref{eq:fQi-factor} substituted into~\eqref{eq:R0-prime} in
the limit \(\Q \to 0\).

Equation~\eqref{eq:parabola-xy-prime-final} can be rewritten in the
form
\begin{gather}
  \label{eq:parabola-xy-all-primes}
  \begin{aligned}
    x_{\T}' &= R_0' \left(1  - \tfrac{1}{2} \Pi' t'^2\right) \\
    y_{\T}' &= R_0' \Pi' t' \ ,
  \end{aligned}
\end{gather}
where
\begin{align}
  \label{eq:parabola-Pi-prime}
  \Pi' &= \frac{2 \Pi} {2 \cos^2 i + \Pi \sin^2 i} \\
  \label{eq:parabola-t-prime}
  t' &= \cos i \left(t^2 - \tan^2 i\right)^{1/2} \ ,
\end{align}
which demonstrates that the projected shape is also a parabola. It is
apparent from \eqref{eq:parabola-Pi-prime} that the projected
planitude obeys
\begin{equation*}
\lim_{i \to 90^\circ} \Pi' = 2 \ ,
\end{equation*}
for all values of the true planitude \(\Pi\), as is shown by the black
lines in Figure~\ref{fig:quadric-projection}a.  Also, for the special
case of the confocal paraboloid, \(\Pi = 2\), we have \(\Pi' = \Pi\) by
equation~\eqref{eq:parabola-Pi-prime} for all inclinations, so its
shape is unaffected by projection. Finally, the projected alatude can
be found as
\begin{equation}
  \label{eq:parabola-Lambda-prime}
  \Lambda' = \left( 2 \Pi' \right)^{1/2} \ .
\end{equation}
%%% Local Variables:
%%% mode: latex
%%% TeX-master: "quadrics-bowshock"
%%% End:
