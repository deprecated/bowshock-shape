\section{Conic section approximation to bow shock shapes}
\label{sec:conic}

%%% Local Variables:
%%% mode: latex
%%% TeX-master: "proplyd-bowshocks"
%%% End:

In this section we will analyze the case where the resultant shape is a conic curve (circle, ellipse, parabola or hyperbola).
These curves are mathematical simple to model and give us a good reference to understand the effects of the projection effects
described in the last section on other bowshocks. Instead of the excentricity, we utilize the parameter $\theta_c$ to characterize the different curves, where
$\tan\theta_c = \frac{b}{a}$,  $b$ and $a$ are the typical parameters of conics. A positive value for $\theta_c$ indicates thethe given curve is a closed one, i.e
an ellipse, while a negative value indicates that is an hyperbola. %Insert figures if neccesary  

Due to the similitudes between the parametrization of the ellipse and the hyperbola, we can do the following parametrization:

\begin{align}
x = au_c(t)-x_0 \\ 
y = bv_c(t)
\end{align}

where:
\begin{align}
u_c(t) = \left\lbrace \begin{array}{c}
\cos t ~\mathrm{if~ellipse} \\
\cosh t ~mathrm{if~hyperbola}
\end{array}\right
\end{align}\\
v_c(t) = \left\lbrace \begin{array}{c}
\sin t ~\mathrm{if~ellipse} \\
\sinh t ~mathrm{if~hyperbola}
\end{array}\right \\
-\pi < t < \pi \\
R_0 = a - x_0 
\end{align}

\subsection{Projection onto the plane of sky} 

Once the parametrization is done, we can find the apparent shape of the shell in the observer's frame, following the procedure explained in section \ref{sec:projection}.

First of all, the intrinsic 3D shape of the shell is given by:

\begin{align}
x = au_c(t)-x_0 \\ 
y = bv_c(t)\cos\phi \\
z =  bv_c(t)\sin\phi
\end{align}

The azimutal angle where the line of sight is tangent to the shell is given by equation (\ref{eq:tanphi}) and (\ref{eq:tanalpha}), then:

\begin{align}
\tan\phi &= \frac{b}{a}w_c(t) 
\end{align}
where:

\begin{align}
w_c(t) = \left\lbrace \begin{array}{c}
\cot t ~\mathrm{if~ellipse} \\
-\coth t ~mathrm{if~hyperbola}
\end{array}\right
\end{align}
\subsection{Characteristic radii}

