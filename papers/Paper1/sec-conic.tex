\section{Conic section approximation to bow shock shapes}
\label{sec:conic}

%%% Local Variables:
%%% mode: latex
%%% TeX-master: "proplyd-bowshocks"
%%% End:

In this section we will analyze the case where the resultant shape is a conic curve (circle, ellipse, parabola or hyperbola).
These curves are mathematical simple to model and give us a good reference to understand the effects of the projection effects
described in the last section on other bowshocks. Instead of the excentricity, we utilize the parameter $\theta_c$ to characterize the different curves, where
$\tan\theta_c = \frac{b}{a}$,  $b$ and $a$ are the typical parameters of conics. A positive value for $\theta_c$ indicates thethe given curve is a closed one, i.e
an ellipse, while a negative value indicates that is an hyperbola. %Insert figures if neccesary  


\subsection{Characteristic radii}

\subsection{Projection onto the plane of sky} 
