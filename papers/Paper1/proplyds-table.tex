 \begin{table*}
\begin{tabular}{|c|c|c|c|c|c|c|c|}
Proplyd & LV2 & LV3 & LV4  & LV5 & 177-341 & 167-328 \\
$R_c/D$ (1) & 0.30 & 0.58  & 0.35 & 0.41 & 0.17 & 0.15 \\
$R_c/D$ (2) & 0.4 & 0.55 & 0.37 & 0.33 & 0.2 & 0.14    \\
$R_c/D$     & 0.35 $\pm$ 0.05 & 0.565 $\pm$ 0.015 & 0.36  $\pm$ 0.01  & 0.37 $\pm$ 0.04 & 0.185 $\pm$ 0.015 & 0.145 $\pm$ 0.005 \\
$R_0/D$ (1) & 0.24 & 0.34 & 0.19 & 0.21 & 0.13 & 0.1 \\
$R_0/D$ (2) & 0.235 & 0.348  & 0.184 & 0.206 & 0.129 & 0.097 \\
$R_0/D$     & 0.238 $\pm$ 0.003 & 0.344 $\pm$ 0.004 & 0.187 $\pm$ 0.003 & 0.208 $\pm$ 0.002 & 0.130 $\pm$ $5\times 10^{-4}$ & 0.099 $\pm$ 0.002 \\
$R_c/R_0$ & 1.471 $\pm$ 0.211 & 1.642 $\pm$ 0.048 & 1.925 $\pm$ 0.062 & 1.779 $\pm$ 0.193 & 1.423 $\pm$ 0.116 & 1.465 $\pm$ 0.059
\end{tabular}
\caption{Characteristic Radii measurements for a sample of proplyds. The (1) mark refers to the fit restricting
the center of the circle to be in the symmetry axis, while (2) refers to the fit without any restriction. The
measurements without any mark were calculated as the mean of the two
fits.}
\label{tab:proplyds}
\end{table*}

%%% Local Variables:
%%% mode: latex
%%% TeX-master: "proplyd-bowshocks.tex"
%%% End:
