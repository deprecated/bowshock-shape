\section{Analytic derivation of thin-shell bow shape parameters}
\label{sec:thin-shell-shapes}

In this appendix, we provide analytic calculations of the planitude,
alatude, and asymptotic opening angle for the wilkinoid, cantoids, and
ancantoids.  We first consider the most general case of the
ancantoids, and then show how results for cantoids and the wilkinoid
follow as special cases.

\subsection{Planitude of ancantoids}
\label{sec:ancantoid-planitude}

From equations~\eqref{eq:radius-curvature} and~\eqref{eq:planitude},
the planitude depends on the apex second derivative,
\(R_{\theta\theta,0}\), as
\begin{equation}
  \label{eq:planitude-from-2nd-derivative}
  \Pi = \left(  1 - R_{\theta\theta,0} / R_0\right)^{-1} \ .
\end{equation}
From equation~\eqref{eq:taylor-R-theta}, the second derivative can be
found from the coefficient of \(\theta^2\) in the Taylor expansion
of \(R(\theta)\).  Since we do not have \(R(\theta)\) in explicit analytic form,
we proceed via a Taylor expansion of the implicit
equations~\eqref{eq:crw-angles}
and~\eqref{eq:ancantoid-theta-theta1-implicit}, retaining terms up to
\(\theta^4\) to obtain from
equation~\eqref{eq:ancantoid-theta-theta1-implicit}:
\begin{equation}
  \label{eq:taylor-expansion-implicit}
  \theta_1^2 = \beta \theta^2 \left( 1 + C_{k\beta} \theta^2\right) + \mathcal{O}(\theta^6)\ , 
\end{equation}
with the coefficient \(C_{k\beta}\) given by
\begin{equation}
  \label{eq:C-k-beta}
  C_{k\beta} = \frac{1}{15} - \frac{3k}{20} - \frac{\beta}{15}  \ .
\end{equation}
Note that it is necessary to include the \(\theta^4\) term in the expansion
of \(\theta_1^2\) so that \(\theta_1/\theta\) is accurate to order
\(\theta^2\).  Then, from equation~\eqref{eq:crw-angles} we find
\begin{align}
  \label{eq:taylor-R-over-D}
  \frac{R}{D} & = \frac{\sin \theta_1} {\sin (\theta + \theta_1)} \nonumber \\
              & = \frac{\beta^{1/2}}{1+\beta^{1/2}}
                \left\lbrace 1 + \theta^2
                \left[ \frac{C_{k\beta}} {2 \left(1+\beta^{1/2}\right)}
                + \frac{1}{6} \left(1+2\beta^{1/2} \right)
                \right]
                \right\rbrace + \mathcal{O}(\theta^4) \ ,
\end{align}
where in the second line we have carried out a Taylor expansion of the
two \(\sin\) terms and substituted
\eqref{eq:taylor-expansion-implicit}.  Comparing coefficients of unity
and \(\theta^2\) between equations~\eqref{eq:taylor-R-theta} and
\eqref{eq:taylor-R-over-D} we find
\begin{align}
  \label{eq:again-R0-over-D}
  \frac{R_0} {D} &= \frac{\beta^{1/2}}{1+\beta^{1/2}} \\
  \label{eq:final-second-derivative}
  \frac{R_{\theta\theta,0}} {R_0} &= \frac{C_{k\beta}}{1+\beta^{1/2}}+\frac{1}{3}\left(1+2\beta^{1/2}\right) \ ,
\end{align}
so that the final result for the planitude, from~\eqref{eq:planitude-from-2nd-derivative}, is
\begin{equation}
  \label{eq:final-planitude}
  \text{ancantoid} \quad
  \Pi = \left[ {1 - \frac{C_{k\beta}}{1+\beta^{1/2}} - \frac{1}{3}\left(1+2\beta^{1/2}\right)}
  \right]^{-1} \ .
\end{equation}

\subsection{Alatude of ancantoids}
\label{sec:ancantoid-alatude}

To find the alatude, \(\Lambda = R_{90} / R_0\), we use equation~\eqref{eq:crw-angles} at \(\theta = 90^\circ\) to write
\begin{equation}
  \label{eq:Lambda-from-theta-1-90}
  \Lambda = \frac{D} {R_0} \tan \theta_{1,90} \ , 
\end{equation}
where \(\theta_{1,90} = \theta_1(\theta = 90^\circ)\), which, following
equation~\eqref{eq:ancantoid-theta-theta1-implicit}, must satisfy
\begin{equation}
  \label{eq:theta-1-90-implicit}
  \theta_{1,90} \cot \theta_{1,90}  = 1 - \frac{2 \beta}{k + 2} \ . 
\end{equation}
Combining \eqref{eq:Lambda-from-theta-1-90} and
\eqref{eq:theta-1-90-implicit} with \eqref{eq:again-R0-over-D} yields
\begin{equation}
  \label{eq:Lambda-beta-xi-theta-1-90}
  \Lambda = \frac{ \left(1 + \beta^{1/2}\right) \,\theta_{1,90}} {\beta^{1/2} \left(1 - \xi_k \beta\right)} \ ,
\end{equation}
where
\begin{equation}
  \label{eq:xi-k}
  \xi_k = \frac{2} {k +2} \ .
\end{equation}
We now take the Taylor expansion of equation~\eqref{eq:theta-1-90-implicit} to find
\begin{equation}
  \label{eq:theta-1-90-Taylor}
  \theta_{1,90}^2 + \tfrac{1}{15}  \theta_{1,90}^4 + \mathcal{O}(\theta_{1,90}^6)
  = 3 \xi_k \beta \ , 
\end{equation}
which, if \(\theta_{1,90}\) is small, has the approximate solution
\begin{equation}
  \label{eq:theta-1-90-approx}
  \theta_{1,90} \approx \left( \frac{3 \xi_k \beta} {1 + \tfrac15 \xi_k \beta} \right) \ .
\end{equation}
Substituting back into equation~\eqref{eq:Lambda-from-theta-1-90}
yields an approximate value for the alatude of
\begin{equation}
  \label{eq:Lambda-approx}
  \text{ancantoid} \quad
  \Lambda \approx \frac {(3 \xi_k)^{1/2} \left( 1 + \beta^{1/2} \right)}
  { \left( 1 + \tfrac15 \xi_k \beta \right)^{1/2} \left( 1 - \xi_k \beta \right)} \ .
\end{equation}
This approximation is surprisingly accurate, with a relative error of
order \(1\%\) even for \(\beta\) as large as \(0.5\) with \(k = 0\).  

\subsection{Planitude and alatude of cantoids and wilkinoid}
\label{sec:cantoid-wilkinoid-shapes}

Since \(\Pi\) and \(\Lambda\) depend on only that portion of the inner wind
emitted in the forward hemisphere, \(\theta \le 90^\circ\), the results for the
cantoids can be found by taking \(k = 0\), in which case
equations~(\ref{eq:C-k-beta}, \ref{eq:final-planitude}, \ref{eq:xi-k},
\ref{eq:Lambda-approx}) yield
\begin{gather}
  \label{eq:cantoid-Pi-Lambda}
  \text{cantoid} \quad
  \begin{cases}
    \quad \Pi &= \dfrac {5} {3 \left( 1 - \beta^{1/2} \right)}\\[12pt]
    \quad \Lambda &= \dfrac { \sqrt 3} {\left( 1 + \tfrac15 \beta \right)^{1/2} \left( 1 - \beta^{1/2} \right)} \ .
  \end{cases}
\end{gather}
The wilkinoid shape is equal to the \(\beta \to 0\) limit of the cantoid, so
its planitude and alatude are given by:
\begin{gather}
  \label{eq:wilkinoid-Pi-Lambda}
  \text{wilkinoid} \quad
  \begin{cases}
    \quad \Pi &= \dfrac {5} {3}\\[10pt]
    \quad \Lambda &= \sqrt 3 \ .
  \end{cases}
\end{gather}
The wilkinoid results can also be obtained directly from
equation~\eqref{eq:wilkinoid-R-theta}, and in the case of \(\Lambda\) this
has already been noted by several authors \citep{Cox:2012a,
  Meyer:2016a}.

\subsection{Asymptotic opening angle}
\label{sec:asympt-open-angle}

The asymptotic opening angle
of the far wings, \(\theta_\infty\), can be found from
equation~\eqref{eq:ancantoid-theta-theta1-implicit} for the
ancantoids, together with the condition that
\(\theta_\infty + \theta_{1\infty} = \pi\).  These yield the implicit equation
\begin{equation}
  \label{eq:ancantoid-theta-inf}
  \theta_\infty - \left( \frac {k + 2 (1 - \beta)} {k + 2} \right) \tan \theta_\infty
  = \pi + 2 \beta I_k(\pi/2) \ ,
\end{equation}
where
\begin{equation}
  I_k(\pi/2) = \frac{\sqrt \pi}{4} \,
      \frac{  \GammaFunc\left( \frac{k+1}{2} \right)} {\GammaFunc\left(\frac{k+4}{2}\right)}
\end{equation}
and \(\GammaFunc\) is the usual Gamma function.  This can be compared with the equivalent result obtained by \CRW{} for the cantoids:
\begin{equation}
  \label{eq:cantoid-theta-inf}
  \theta_\infty - \tan \theta_\infty = \frac{\pi}{1 - \beta} \ .
\end{equation}
Note that, unlike in the cases of \(\Pi\) and \(\Lambda\),
equation~\eqref{eq:ancantoid-theta-inf} does \emph{not} reduce to
equation~\eqref{eq:cantoid-theta-inf} in the limit \(k \to 0\).  This is
because, for \(\theta > 90^\circ\), the \(k = 0\) ancantoid differs from the
cantoid since the former has no wind in the backward hemisphere (see
Figure~\ref{fig:anisotropic-arrows}).  Therefore there is less inner
support for the far wings of the bow, and so \(\theta_\infty\) is smaller than
in the cantoid case.  The wilkinoid result again follows from
\(\beta \to 0\), implying that \(\theta_\infty = \pi\), or, in other words, that the far
wings are asymptotically parallel to the symmetry axis, as is the case
for the paraboloid (App.~\ref{app:parabola}).  In the case of the
wilkinoid, however, the behavior is cubic in the wings,
\(z \sim r^3\), as opposed to quadratic as in the paraboloid.
%%% Local Variables:
%%% mode: latex
%%% TeX-master: "quadrics-bowshock"
%%% End:
