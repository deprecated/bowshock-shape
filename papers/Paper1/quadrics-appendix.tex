\appendix
% \appendixpage
% \addappheadtotoc
\section{Parabola of Revolution}
\label{app:parabola}

In this section we develop the particular case of the parabola.
The parametrization is given by:

\begin{align}
x &= - \frac{1}{2}R_ct^2 + R_0 \\
y &= R_c t
\end{align}
Where $R_c$ is the radius of curvature and $R_0$ is the distance of the parabola nose to the
origin.
The respective tridimensional shape is given by:
\begin{align}
x &= -\frac{1}{2}R_ct^2 + R_0 \label{eq:x-par-a}\\
y &= R_c t \cos\phi  \label{eq:y-par-a}\\
z &= R_c t \sin\phi  \label{eq:z-par-a}
\end{align}
The azimutal angle where the surface is tangent to the line of sight in this case is given by:
\begin{align}
\sin\phi_t = -\frac{\tan i}{t} \label{eq:sin-tan-a} 
\end{align}

Subtituting (\ref{eq:sin-tan-a}) into (\ref{eq:x-par-a}), (\ref{eq:y-par-a}) and
(\ref{eq:z-par-a}) we find the apparent shape
of the paraboloid:

\begin{align}
x' &= -\frac{R_c(\frac{1}{2}t^2 \cos^2 i -\sin^2 i)}{\cos i}+R_0\cos i \\
y' &= R_c\left(t^2-\tan^2 i\right)^{1/2} 
\end{align}

Taking the limit of equations (\ref{eq:qprime}) and (\ref{eq:Aprime}) when $\theta_c$ tends to zero we find that:

\begin{align}
\left(\frac{q'}{q}\right)_{\mathrm{parabola}} &= 1+\frac{A\tan^2 i}{2}\\
\left(A'\right)_{\mathrm{parabola}} &= \frac{A}{\cos^2 i + \frac{A}{2}\sin^2 i}
\end{align}

\section{Analytic derivation of the radius of curvature in the Thin Shell model}
\label{app:rc-analytic}

For small $\theta$ we may do a polynomial expansion for the shell shape such as:

\begin{align}
R \simeq R_0\left(1+\gamma\theta^2 + \Gamma\theta^4\right) \label{eq:R-exp}
\end{align}

The radius of curvature at the axis for $R$ is given by:

\begin{align}
R_c = R_0\left(1-2\gamma\right)^{-1}
\end{align}

The coefficient gamma may be derived by an expansion at small angles of equation
(\ref{eq:th1th}), as follows:

From the first term of the right side we get:

\begin{align}
\cot\theta &\simeq \theta^{-1}\left[1-\frac{1}{3}\theta^2\right] \\
\cos^k\theta\sin^2\theta &\simeq \theta^2 - \left(\frac{1}{3} + \frac{k}{2}\right)\theta^4 \\
\implies I_k(\theta) &\simeq \frac{1}{3}\theta^3\left[ 1 - \frac{1}{10}(3k+2)\theta^2\right]\\
\implies 2\beta I_k(\theta)\cot\theta &\simeq \frac{2}{3}\beta\theta^2\left[1-\frac{1}{30}
(9k+16)\theta^2\right]\label{eq:AR1} 
\end{align}

For the second term we get:

\begin{align}
-\frac{2\beta}{k+2}\left(1-\cos^{k+2}\theta\right) & \simeq -\beta\theta^2\left[1-\frac{1}{12}
(3k+4)\theta^2\right] \label{eq:AR2}
\end{align}

For the left side we use equation (25) from \CRW{}. Then, equation (\ref{eq:th1th}) results
as follows:

\begin{align}
\theta_1^2\left(1+\frac{1}{15}\theta_1^2\right) = \beta\theta^2\left[1+\frac{1}{60}(4-9k)
\theta^2\right] \label{eq:th1th-app}
\end{align}

And we can use the approximation $\theta_1 \approx \beta\theta^2$ for the correction term in
the left side of (\ref{eq:th1th-app}):

\begin{align}
\theta_1^2 &= \beta\theta^2\left[1+\frac{1}{60}(4-9k)\theta^2\right]
\left(1-\frac{\beta}{15}\theta^2\right) \\
\implies \theta_1^2 &= \beta\theta^2\left[1+ 2C_{k\beta}\theta^2\right]
\label{eq:th1th-small}\\
\mathrm{where:~} C_{k\beta} &\equiv \frac{1}{2}\left(A_k-\frac{\beta}{15}\right) \\
A_k &\equiv \frac{1}{15}-\frac{3k}{20}
\end{align}

Now, using equation (23) from \CRW{} we may estimate $R$ at low angles. To do this, we need to
expand each term as follows (neglecting terms of order four or higher):

\begin{align}
\theta_1 = &= \beta^{1/2}\theta\left[1+ 2C_{k\beta}\theta^2\right]^{1/2} \\
\theta + \theta_1 &= \theta\left[1+\beta^{1/2}\left(1+2C_{k\beta}\theta^2\right)\right]\\
\sin\theta_1 &= \theta_1\left[1-\frac{\theta_1^2}{6}\right] \\
 &= \beta^{1/2}\theta\left[1+\left(C_{k\beta}-\frac{1}{6}\beta\right)\theta^2\right] \\
 \sin(\theta+\theta_1) &= \left[\theta+\theta_1\right]\left[1-\frac{\left(\theta+\theta_1
 \right)^2}{6}\right] \\
 &= \theta\left(1+\beta^{1/2}\right)\left\lbrace 1+\left[\frac{C_{k\beta}\beta^{1/2}}
 {1+\beta^{1/2}}-\frac{1}{6}\left(1+\beta^{1/2}\right)^2\right]\theta^2\right\rbrace
\end{align}


So, combining these terms with equation (23) from \CRW{} we found the final expression for $R$:

\begin{align}
\frac{R}{D}\equiv \frac{\sin\theta_1}{\sin(\theta+\theta_1)} = \frac{\beta^1/2}{1+\beta^{1/2}}
\left\lbrace 1 + \theta^2\left[\frac{C_{k\beta}}{1+\beta^{1/2}}+\frac{1}{6}\left(1+2\beta^{1/2}
\right)\right] \right\rbrace \label{eq:r-small-theta}
\end{align}

Returning to equation (\ref{eq:R-exp}) we see the following:

\begin{align}
R_0 &= \frac{\beta^{1/2}}{1+\beta^{1/2}} \\
\gamma &= \frac{C_{k\beta}}{1+\beta^{1/2}}+\frac{1}{6}\left(1+2\beta^{1/2}\right)
\label{eq:app-gamma}
\end{align}

We recover equation (27) of \CRW{} for $R_0$ and equation (\ref{eq:app-gamma}) is the
needed term to calculate the radius of curvature at the axis.

\section{Analytic derivation of \texorpdfstring{\boldmath $R_{90}$}{R\_90} in the thin shell model}
\label{app:r90-analytic}

To derive $R_{90}$ we need to evaluate equations (23) from \CRW{} and (\ref{eq:th1th})
at $\theta=\frac{\pi}{2}$:

\begin{align}
R_{90} = D\tan\theta_{1,90} \\
\theta_{1,90}\cot\theta_{1,90} -1 = -\frac{2\beta}{k+2} \label{eq:th190}
\end{align}
Where $\theta_{1,90}\equiv \theta_1(\frac{\pi}{2})$. Combining both equations and  introducing
the parameter $\xi\equiv \frac{2}{k+2}$ we have:
\begin{align}
R_{90} &= D\frac{\theta_{1,90}}{1-\xi\beta} \label{eq:r90-incomplete}
\end{align}

Expanding the left side of (\ref{eq:th190}) until fourth order, equation (\ref{eq:th190})
becomes:

\begin{align}
\theta_{1,90}^2\left(1+\frac{\theta_{1,90}^2}{15}\right) \simeq 3\xi\beta
\end{align}

Applying the approximation $\theta_1^2 \approx 3\xi\beta$ we found a solution
for $\theta_{1,90}$:

\begin{align}
\theta_{1,90} = \left(\frac{3\xi\beta}{1+\frac{1}{5}\xi\beta}\right)^{1/2}
\end{align}

And substituting into (\ref{eq:r90-incomplete}) we find the solution for $R_{90}$:

\begin{align}
R_{90} &= \frac{\left(3\xi\beta\right)^{1/2}}{\left(1+\frac{1}{5}\xi\beta\right)^{1/2}
\left(1-\xi\beta\right)} \\
\implies B &\equiv \frac{R_{90}}{R_0} = \frac{\sqrt{3\xi}\left(1+\beta^{1/2}\right)}
{\left(1+\frac{1}{5}\xi\beta\right)^{1/2}\left(1-\xi\beta\right)}
\end{align}

%%% Local Variables:
%%% mode: latex
%%% TeX-master: quadrics-bowshock
%%% End:
