\appendix
\appendixpage
\addappheadtotoc
\section{Parabola of Revolution}
\label{app:parabola}

In this section we develop the particular case of the parabola.
The parametrization is given by:

\begin{align}
x &= - \frac{1}{2}R_ct^2 + R_0 \\
y &= R_c t
\end{align}
Where $R_c$ is the radius of curvature and $R_0$ is the distance of the parabola nose to the
origin.
The respective tridimensional shape is given by:
\begin{align}
x &= -\frac{1}{2}R_ct^2 + R_0 \label{eq:x-par-a}\\
y &= R_c t \cos\phi  \label{eq:y-par-a}\\
z &= R_c t \sin\phi  \label{eq:z-par-a}
\end{align}
The azimutal angle where the surface is tangent to the line of sight in this case is given by:
\begin{align}
\sin\phi_t = -\frac{\tan i}{t} \label{eq:sin-tan-a} 
\end{align}

Subtituting (\ref{eq:sin-tan-a}) into (\ref{eq:x-par-a}), (\ref{eq:y-par-a}) and
(\ref{eq:z-par-a}) we find the apparent shape
of the paraboloid:

\begin{align}
x' &= -\frac{R_c(\frac{1}{2}t^2 \cos^2 i -\sin^2 i)}{\cos i}+R_0\cos i \\
y' &= R_c\left(t^2-\tan^2 i\right)^{1/2} 
\end{align}

Taking the limit of equations (\ref{eq:qprime}) and (\ref{eq:Aprime}) when $\theta_c$ tends to zero we find that:

\begin{align}
\left(\frac{q'}{q}\right)_{parabola} &= 1+\frac{A\tan^2 i}{2}\\
\left(A'\right)_{parabola} &= \frac{A}{\cos^2 i + \frac{A}{2}\sin^2 i}
\end{align}

\section{Analytic derivation of the raius of curvature in the Thin Shell model}
\label{app:rc-r90-analytic}

For small $\theta$ we may do a polynomial expansion for the shell shape such as:

\begin{align}
R \simeq R_0\left(1+\gamma\theta^2 + \Gamma\theta^4\right) \label{eq:R-exp}
\end{align}

With this, the radius of curvature at the axis for $R$ is given by equation (\ref{eq:Rcurv}).

The coefficient gamma may be derived by an expansion at small angles of equation
(\ref{eq:th1th}), as follows:

From the first term of the right side we get:

\begin{align}
\cot\theta &\simeq \theta^{-1}\left[1-\frac{1}{3}\theta^2\right] \\
\cos^k\theta\sin^2\theta &\simeq \theta^2 - \left(\frac{1}{3} + \frac{k}{2}\right)\theta^4 \\
\implies I_k(\theta) &\simeq \frac{1}{3}\theta^3\left[ 1 - \frac{1}{10}(3k+2)\theta^2\right]\\
\implies 2\beta I_k(\theta)\cot\theta &\simeq \frac{2}{3}\beta\theta^2\left[1-\frac{1}{30}
(9k+16)\theta^2\right]\label{eq:AR1} 
\end{align}

For the second term we get:

\begin{align}
-\frac{2\beta}{k+2}\left(1-\cos^{k+2}\theta\right) & \simeq -\beta\theta^2\left[1-\frac{1}{12}
(3k+4)\theta^2\right] \label{eq:AR2}
\end{align}

For the left side we use equation (25) from \CRW{}. Then, equation (\ref{eq:th1th}) results
as follows:

\begin{align}
\theta_1^2\left(1+\frac{1}{15}\theta_1^2\right) = \beta\theta^2\left[1+\frac{1}{60}(4-9k)
\theta^2\right] \label{eq:th1th-app}
\end{align}

And we can use the approximation $\theta_1 \approx \beta\theta^2$ for the correction term in
the left side of (\ref{eq:th1th-app}):

\begin{align}
\theta_1^2 &= \beta\theta^2\left[1+\frac{1}{60}(4-9k)\theta^2\right]
\left(1-\frac{\beta}{15}\theta^2\right) \\
\implies \theta_1^2 &= \beta\theta^2\left[1+ \left(A_k-\frac{\beta}{15}\right)\theta^2\right]\\
\mathrm{where:~} A_k &\equiv \frac{1}{15}-\frac{3k}{20}
\end{align}

