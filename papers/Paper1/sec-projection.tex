\section{Projection onto the plane of the sky}
\label{sec:projection}

In this section we calculate the apparent shape of a limb brightened  border of a generic bow shock with cilindrical geometry, observed in the
plane of the sky.

%Note: I'm aware that some of this material should be moved to an appendix, but I think it will be a future edition.
\subsection{Frames of reference}

We can specify the shell's position in cartesian coordinates $(x,y,z)$, where the x axis points towards the symetry axis of the system.
Since the shell is cilindrically symetric, we can describe it's shape fully with the function $R(\theta)$, such that:

\begin{equation}
\left(\begin{array}{c}
x \\ y \\ z
\end{array}
\right) = R(\theta)\left(\begin{array}{c}
\cos\theta \\
\sin\theta\cos\phi \\
\sin\theta\sin\phi
\end{array}\right)
\end{equation} 

Where $\theta$ is the polar angle and $\phi$ the azimutal angle from spherical coordinates

The observer's reference frame is denotated with primes, and is rotated respect y axis by an angle $i$. The transformation between the shell frame and the observed
frame is given by:

\begin{equation}
\left(\begin{array}{c}
x' \\ y' \\ z'
\end{array}
\right) = \left(\begin{array}{c}
x\cos i - z\sin i\\
y \\
z\cos i + x\sin i
\end{array}\right)
\label{eq:Trans}
\end{equation} 

\subsection{Tangent line}

The normal and tangent vectors to the shell's border in the shell's frame are given by:

\begin{align}
\hat{t} = \left(\begin{array}{c}
-\cos\alpha \\
\sin\alpha\cos\phi\\
\sin\alpha\sin\phi
\end{array}\right)\\
\hat{n} = \left(\begin{array}{c}
\sin\alpha \\
\cos\alpha\cos\phi \\
\cos\alpha\sin\phi
\end{array}\right)
\end{align}

Where:

\begin{align}
\tan\alpha = -\left\frac{dy}{dx}\right|_{R(\theta)} \\
\label{eq:tanalpha}
\end{align}

Then we apply the transformation (\ref{eq:Trans}) to the normal and tangent vectors to obtain:

\begin{align}
\hat{n}' &= \left(\begin{array}{c}
(\cos\theta+\omega\sin\theta)\cos i -(\sin\theta-\omega\cos\theta)\sin i \sin\phi\\
(\sin\theta-\omega\cos\theta)\cos\phi \\
(\cos\theta+\omega\sin\theta)\sin i + (\sin\theta-\omega\cos\theta)\sin\phi\cos i
\end{array}\right)\\
\hat{t}' &= \left(\begin{array}{c}
-(\sin\theta-\omega\cos\theta)\cos i - (\cos\theta+\omega\sin\theta)\sin\phi\sin i \\
(\cos\theta+\omega\sin\theta)\cos\phi \\
-(\cos\theta+\omega\sin\theta)\sin i + (\sin\theta-\omega\cos\theta)\sin\phi\cos i
\end{array}\right)
\end{align}

Where $\omega = \frac{1}{R}\frac{dR}{d\theta}$

In the thin shell case, the limb brightened border of the shell is such that $\hat{n}'\cdot \hat{z}'$. 
The values for $\phi$ that satisfy this relation for each inclination $i$ are given by:

\begin{equation}
\sin\phi_t = \tan i\tan\alpha = \tan i \frac{1+\omega\tan\theta}{\omega-\tan\theta}
\label{eq:tanphi}
\end{equation}

With this, the coordinates of the limb brightened shell are given by:

\begin{equation}
\left(\begin{array}{c}
x'_t \\ y'_t \\ z'_t
\end{array}\right)= R(\theta)\left(\begin{array}{c}
\cos\theta\cos i - \sin\theta\sin\phi_t \sin i \\
\sin\theta(1-\sin^2\phi_t)^{1/2} \\
\cos\theta\sin i +\sin\theta\sin\phi_t\cos i
\end{array}\right)
\label{eq:tangential}
\end{equation} 

It is important to note that the equation (\ref{eq:tanphi}) does not have a solution for arbitrary values for $\theta$ and $i$, since
it's required that $|\sin\phi_t|<1$. If $i\neq 0$, then, the  allowed values for $\theta$ are such that $\theta > \theta_\parallel$, where
$\theta_\parallel$ is given implicitly by:

\begin{align}
\tan\theta_\parallel = \frac{|\tan i| + \omega(\theta_\parallel)}{1-\omega(\theta_\parallel |\tan i|)}
\label{eq:thetapar}
\end{align}

\subsection{Parallel and perpendicular projected shell radii}

Considering further applications to bow shocks, we will consider open shells. In order to compare the shell shape given by $R(\theta)$ with observations,
it is convenient to define the following apparent radii in the observer frame: $R'_\parallel$ and $R'_\perp$. These are projected distances of the shell tangent line
from the origin. The first is measured in the direction of the symetry axis, and the second in a perpendicular direction. More concretely $R'_\parallel = x'_t(y'_t=0)$
and $R'_\perp = y'_t(x'_t=0)$. From equations (\ref{eq:tanphi}) and (\ref{eq:tangential}) we find that:

\begin{align}
R'_\parallel = R(\theta_\parallel)\cos(\theta + i) \label{eq:Rpar} 
\end{align}

Where $\theta_\parallel$ is the solution of equation (\ref{eq:thetapar}), and

\begin{align}
R'_\perp = R(\theta_\perp)\sin\theta_\perp\left(1-\sin^2(\phi_t(\theta_\perp))\right)^{1/2}
\end{align}

Where $\theta_\perp$ is the solution of the next implicit equation:

\begin{align}
\cot\theta_\perp = \frac{1-\left(1+\omega(\theta_\perp)^2\sin^22i\right)^{1/2}}{2\omega(\theta_\perp\cos^2 i)}
\end{align}


%%% Local Variables:
%%% mode: latex
%%% TeX-master: "proplyd-bowshocks"
%%% End:

