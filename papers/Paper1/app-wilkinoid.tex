\section{Derivation of Characteristic Radii in Isotropic Wind/Parallel interaction Problem}
\label{app:ch-rad-Wilkin}

$\tilde{R}_{90}$ is obtained by simply evaluating equation (\ref{eq:R-Wilkin}) at $\theta=\frac{\pi}{2}$.
For the Radius of curvature we follow a similar procedure than the Wind/Wind interaction, but using equation
(\ref{eq:R-Wilkin}) for $R(\theta)$ and inserting the cosecant into the square root.

Expanding the terms of $R(\theta)$ we find the following:

\begin{align}
  \csc^2\theta &\simeq \theta^{-2}\left[1+\frac{\theta^2}{3}\left(1+\frac{\theta^2}{5}\right)\right] \\
  1-\theta\cot\theta &\simeq \frac{\theta^2}{3}\left[1 + \frac{\theta^2}{15}\left(1+\frac{2\theta^2}{21}\right)\right] \\
  \implies \tilde{R}(\theta) \simeq 1 + \frac{\theta^2}{5} + O\left[\theta^4\right]
\end{align}

From equation (\ref{eq:Rcurv}) for the radius of curvature we finally get the numerical value for $\tilde{R_c} = \frac{5}{3}$


%%% Local Variables:
%%% mode: latex
%%% TeX-master: "quadrics-bowshock"
%%% End:
