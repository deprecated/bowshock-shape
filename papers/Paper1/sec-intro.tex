\section{Introduction}
\label{sec:intro}

%%
%% Circumstances when bowshocks arise
%%

The archetypal bow shock is formed when a solid body moves
supersonically through a compressible fluid.  Terrestrial examples
include the atmospheric re-entry of a space capsule, or the sonic boom
produced by a supersonic jet \citep{van-Dyke:1982a}.  In astrophysics
the term bow shock is employed more widely, to refer to many different
types of curved shocks that have approximate cylindrical symmetry.
Instead of a solid body, astrophysical examples usually involve the
interaction of \emph{two} supersonic flows, such as the situation of a
stellar wind emitted by a star that moves supersonically through the
interstellar medium \citep{}.  In such cases, two shocks are generally
produced, one in each flow, and technically speaking only the shock in
the ambient medium is a ``bow shock'', whereas the other is the ``wind
shock'' or ``termination shock''.  However, 


%%
%% Examples of astrophysical bowshocks
%%

%% 
%% Restriction to cylindrical symmetry
%% 

For simplicity, the current paper is restricted to cylindrically
symmetric bowshock shapes. 

%%
%% Effects of instabilities
%%


%%% Local Variables:
%%% mode: latex
%%% TeX-master: "proplyd-bowshocks"
%%% End:
